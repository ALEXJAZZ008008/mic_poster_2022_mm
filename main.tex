\documentclass[portrait, color=UCLburgundy, margin=1cm]{uclposter}

\linespread{1.0}

\input{commands.tex}

\usepackage{bm}
\usepackage{algorithm}
\usepackage{algorithmic}
\usepackage{caption}
\usepackage{blindtext}
\usepackage{siunitx}

\input{glossary}

\usepackage[style=ieee, maxbibnames=1, minbibnames=1, maxcitenames=1, mincitenames=1, backend=biber, defernumbers=false]{biblatex}
\addbibresource{./Biblio.bib}

\AtEveryBibitem{\clearfield{month}}
\AtEveryBibitem{\clearfield{day}}
\AtEveryBibitem{\clearfield{volume}}
\AtEveryBibitem{\clearfield{issue}}
\AtEveryBibitem{\clearfield{pages}}
\AtEveryBibitem{\clearfield{number}}
\AtEveryBibitem{\clearfield{title}}
\AtEveryBibitem{\clearfield{isbn}}
\AtEveryBibitem{\clearfield{keywords}}
\AtEveryBibitem{\clearfield{issn}}
\AtEveryBibitem{\clearfield{journal}}

\usepackage{fontspec}
\setmainfont[Ligatures=TeX]{LexendDeca-Regular.ttf}

\begin{document}
    \title{PET/CT Motion Correction Exploiting Motion Models Fit \newline~on Coarsely Gated Data Applied to Finely Gated Data}
    
    \author[12*]{Alexander~C.~Whitehead}
    \author[3]{Kuan-Hao~Su}
    \author[3]{Scott~D.~Wollenweber}
    \author[2]{\newline~Jamie~R.~McClelland}
    \author[12]{Kris~Thielemans}
    
    \affil[1]{INM, UCL}
    \affil[2]{CMIC, UCL}
    \affil[3]{GE Healthcare}
    \affil[*]{alexander.whitehead.18@ucl.ac.uk}
    
    \maketitle

    \begin{multicols}{2}
        \section*{Introduction}
            \begin{highlightbox}[UCLlightgreen]
                \begin{itemize}
                    \item \glss{MM} parameterise \glss{DVF} in terms of a \gls{SS}.
                    \item \glss{MM} can obtain \glss{DVF} for unseen data~\cite{McClelland2013}.
                    \item Previous work~\cite{Whitehead2021ComparisonMap} indicated that \glss{MM} and \acrshort{TOF} increase resolution.
                    \item This work:
                    \begin{itemize}
                        \item Incorporates \acrshort{MLACF}~\cite{Nuyts2012ML-reconstructionFactors}.
                        \item Incorporates a diffeomorphic velocity field parameterised registration.
                        \item Fits the \gls{MM} on coarsely gated data and applies it to finely gated data.
                        \item Uses more realistic simulation and count levels.
                        \item Differentiates itself by using two \glss{SS}, and group-wise registration.
                    \end{itemize}
                \end{itemize}
            \end{highlightbox}
            
        \subsection*{\underline{\textbf{Evaluation}}}
            \begin{itemize}
                \item \gls{MC} was also applied to data in the same way, but using high noise high temporal/gate resolution, or noiseless data, for the \gls{MM} fitting.
                \item Data also reconstructed without \gls{MC}, using either a sum of all \acrshort{Mu-Map} or the end inhalation \acrshort{Mu-Map}.
                \item Volumes without \gls{MC} registered to the position of the \acrshort{Mu-Map}.
                \item \glss{DVF} generated by each method were also applied to noiseless data for visual analysis.
                \item Comparisons used included: A visual analysis, \acrshort{SSIM} to the ground truth~\cite{Wang2009MeanMeasures}, a profile over the lesion, \acrshort{SUV}\textsubscript{max} and \acrshort{SUV}\textsubscript{peak}.
            \end{itemize}
    \end{multicols}
    
    \begin{figure}[H]
        \centering
        
        \includegraphics[width=1.0\linewidth]{visual_analysis.png}
        
        \begin{highlightbox}[UCLlightblue]
            \captionsetup{singlelinecheck=false, justification=centering}
            \caption{First row reconstructions with \acrshort{AC} and \gls{MC}, second row noiseless data. Colour map ranges consistent for all images in each column.}
        \end{highlightbox}
        
        \label{fig:visual_analysis}
    \end{figure}
    
    \begin{multicols}{2}
        \section*{Methods}
            \subsection*{\underline{\textbf{\acrshort{XCAT} Volume Generation}}}
                \begin{itemize}
                    \item \acrshort{XCAT} generated 480 volumes using a 240s respiratory trace.
                    \item \acrshort{FOV} including the base of the lungs with a 20mm diameter lesion.
                \end{itemize}
            
            \subsection*{\underline{\textbf{\acrshort{PET} Acquisition Simulation}}}
                \begin{itemize}
                    \item Simulated corresponding to a \acrshort{GE} Discovery 710.
                    \item Pseudo-randoms and scatter were added.
                    \item Noise was simulated, such that data matched an acquisition over 240s. The count rate was selected to match that of research scans.
                    \item A respiratory \gls{SS} was generated using \acrshort{PCA}~\cite{Thielemans2011}.
                    \item Gated into 4 respiratory bins using the \gls{SS} and its gradient, each bin was a quadrant centred on the maximum or minimum of the displacement or gradient.
                \end{itemize}
            
            \subsection*{\underline{\textbf{\acrshort{MLACF} Image Reconstruction}}}
                \begin{itemize}
                    \item \acrshort{MLACF} (7 full iterations, 24 subsets) for the activity update, and 9 iterations for the attenuation update~\cite{Nuyts2012ML-reconstructionFactors}.
                    \item Initialised using 1 iteration of \acrshort{MLEM}, with breath hold \acrshort{CT} for \acrshort{AC}.
                    \item Normalised between iterations and epsilon added.
                    \item Quadratic prior on emission image.
                \end{itemize}
            
            \subsection*{\underline{\textbf{Registration}}}
                \begin{itemize}
                    \item Pre-processing including; replication of end-slices, smoothing, and\newline~standardisation.
                    \item Between each iteration, resampled volume was registered to the \acrshort{Mu-Map} and \glss{DVF} composed together.
                \end{itemize}
            
            \subsection*{\underline{\textbf{\acrlong{MM} Estimation}}}
                \begin{itemize}
                    \item \gls{MM} fit using weighted \acrlong{LR} between registration \glss{DVF} and 2 \glss{SS}.
                    \item \gls{MM} fit between each iteration.
                \end{itemize}
            
            \subsection*{\underline{\textbf{Image Reconstruction with \acrshort{AC}}}}
                \begin{itemize}
                    \item Re-gated into 30 respiratory bins using displacement gating, \gls{SS} and its gradient (10 amplitude and 3 gradient bins).
                    \item \acrshort{Mu-Map} determined using the inverse of the \glss{DVF} from the \gls{MM}.
                    \item Re-reconstructed with \acrshort{AC} using \acrshort{OSEM} (2 full iterations, 24 subsets).
                    \item Volumes post-filtered with a Gaussian smoothing, (\acrshort{FWHM} of 6.4mm in transverse plane and a normal Z-filter).
                \end{itemize}
        
        \begin{figure}[H]
            \centering
            
            \includegraphics[width=1.0\linewidth]{profile.png}
            
            \begin{highlightbox}[UCLlightblue]
                \captionsetup{singlelinecheck=false, justification=centering}
                \caption{Profile across the lesion.}
            \end{highlightbox}
            
            \label{fig:profile}
        \end{figure}
        
        \begin{table}[H]
            \centering
            
            \begin{highlightbox}[UCLlightblue]
                \captionsetup{singlelinecheck=false, justification=centering}
                \caption{Comparison of \acrshort{SUV}\textsubscript{max} and \acrshort{SUV}\textsubscript{peak}.}
            \end{highlightbox}
            
            \vspace{1.0cm}
            
            \resizebox*{1.0\linewidth}{!}
            {
                \begin{tabular}{||c|cc||}
                    \hline
                    \textbf{\acrshort{SUV}}                 & \textbf{Max}  & \textbf{Peak} \\
                    \hline
                    \textbf{Ground Truth}                   & 8.76          & 7.96 \\
                    \hline
                    \textbf{4 Gate \gls{MM}}                & 8.04          & 6.18 \\
                    \textbf{30 Gate \gls{MM}}               & 1.77          & 1.32 \\
                    \hline
                    \textbf{4 Gate Noiseless \gls{MM}}      & 8.05          & 6.24 \\
                    \textbf{30 Gate Noiseless \gls{MM}}     & 7.96          & 5.99 \\
                    \hline
                    \textbf{Ungated, Static \acrshort{CT}}  & 6.61          & 5.08 \\
                    \textbf{Ungated, \acrlong{AV-CCT}}      & 5.65          & 4.44 \\
                    \hline
                \end{tabular}
            }
            \label{tab:suv}
        \end{table}
        
        \section*{Conclusion}
            \begin{highlightbox}[UCLlightgreen]
                \begin{itemize}
                    \item A low number of gates for \gls{MM} fitting has minimal impact at low noise and improves \gls{MC} when there is a high level of noise in the gates
                    \item In the future, work will focus on evaluating the method on patient data.
                \end{itemize}
            \end{highlightbox}
        
        \AtNextBibliography{\small}
        \printbibliography
    \end{multicols}
\end{document}
